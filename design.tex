
\documentclass[11pt]{article}

\usepackage{amsmath, amssymb}
\usepackage{natbib}
\usepackage{url}

\renewcommand{\mod}{\mathbin{\rm mod}}
\newcommand{\opor}{\mathbin{\rm or}}
\newcommand{\opand}{\mathbin{\rm and}}

\DeclareMathOperator{\logit}{logit}

\newcommand{\abs}[1]{\lvert #1 \rvert}

\begin{document}

\title{Design of molevelset Package}

\author{Leif Johnson}

\maketitle

\section{Introduction}

The purpose of this package is to provide a reasonably fast
implementation of the minimax optimal level set estimation method set
forth in \cite{willet-nowak}.  A reference implementation is available
for Matlab, \cite{willet-code}, but I use \emph{R}.

If we have a function $f:\mathbb{R}^d\mapsto \mathbb{R}$, the level
set of $f$ is the domain for which $f$ exceeds a threshold.  In the
notation of \cite{willet-nowak} $\gamma$ is used to denote the
threshold, and $S_\gamma$ (or sometimes just $S$) is used to denote
the level set, thus
\[
S_\gamma = \{\: x \in \mathbb{R}^d \: | \: f(x) \geq \gamma \: \}
\]
Of course we are assuming that instead of $f(x)$, we are observing 
\[
y = f(x) + \epsilon
\]
where $\epsilon$ is mean zero iid noise.  For a complete
description/discussion, see \cite{willet-nowak}.

\section{Usage}
The goal is to provide a function for \emph{R}, that a user could call
with a form as simple as 
\[
\texttt{l.set <- molevelset(X, Y, gamma)}
\]
and get back some form of the level set estimation.  Secondly, we will
provide another function that will allow the user to convert the level
set estimation to a grid of points estimated to be on the boundary of
the level set e.g.
\[
\texttt{Xp <- boundary.points(molevelset(X, Y, gamma))}
\]

Optionally, $Y$ may be left \texttt{NULL}, and the last column of $X$
will be used as $Y$.  For simplicities sake, this document assumes
that it has not been left \texttt{NULL}.  Internally, $X$ is expected
to be inside of the region $[0,1]^d$, but we will not burden the user
with this, the function will automatically translate to $[0,1]^d$ and
the level set back to the range of $X$.

There are two parameters that control the behavior of the level set
estimation, $k_{max}$ and $\rho$.  $k_{max}$ is an upper limit on the
number of splits (in any of the components of the domain) that can be
used in the tree estimation, and $\rho$ is a penalty factor for tree
complexity.  $k_{max}$ specifies where the domain of $f$ can be
split.  The method uses the splits 
\[
\{ \: \frac{i}{2^{k_{max}}} \: | \: 1 \leq i \leq 2^{k_{max}} - 1 \: \}
\]
$\rho=0$ means no penalty for tree complexity, and $\rho=1$ would
return a tree with very large boxes (possibly either of the degenerate
trees, the whole space or the empty set).  Hence the user must be able
to optionally control the values of these parameters.  No other
parameters to the function need to be exposed externally.

\begin{thebibliography}{}



\bibitem[Willet and Nowak(2007)]{willet-nowak}
  Willett, R.,  and Nowak, R. (2007).
\newblock ``Minimax Optimal Level Set Estimation.'' 
\newblock \emph{IEEE Transactions on Image Processing}, \textbf{16}, 2965--2979.

\bibitem[Willet(2007)]{willet-code}
  Willet, R. (2007).
\newblock ``Level set estimation toolbox.''  
\newblock \url{http://people.ee.duke.edu/~willett/Research/levelset.html}
\end{thebibliography}

\end{document}

